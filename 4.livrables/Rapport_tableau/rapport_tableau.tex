\documentclass[a4paper,12pt]{report}

\usepackage[lmargin=2.00000cm,rmargin=2.0000cm,tmargin=5.5000cm,bmargin=2.500000cm,headheight=3.5cm]{geometry}        %Flexible and complete interface to document dimensions
\usepackage[utf8]{inputenc}
\usepackage[T1]{fontenc}
%\usepackage[latin1]{inputenc}
\usepackage{amsmath}
\usepackage{amsfonts}
\usepackage{amssymb}
\usepackage{lmodern}
\usepackage{float}
\usepackage{graphicx}

\usepackage[english,french]{babel}        %use for the below package `datetime'
\usepackage[babel=true]{csquotes}
\usepackage{datetime}        %Change format of `\today' with commands for current time
\renewcommand{\dateseparator}{-}
\newcommand{\headertoday}{\twodigit\day \dateseparator \twodigit\month  \dateseparator \the\year}

\usepackage{animate}
\usepackage{lastpage}
\usepackage{array}
\usepackage{lastpage}
\usepackage{multirow}
\usepackage{titling}
\usepackage{placeins}
\usepackage{media9}
\usepackage{eurosym}
\usepackage{pdflscape}
\usepackage{color}
\usepackage[table]{xcolor}
\definecolor{mon_bleu}{rgb}{0.137,0.466,0.741}
\definecolor{mon_vert}{rgb}{0.07843,0.4627,0.07843}
\definecolor{mon_rouge}{rgb}{0.62745,0.16078,0.27058}		
%\usepackage{subfigure}
\usepackage{subcaption}

\usepackage[ 
           hidelinks,
           colorlinks=true,
           linkcolor=blue,          % color of internal links (change box color with linkbordercolor)
    	   citecolor=blue,        % color of links to bibliography
    	   filecolor=blue,      % color of file links
    	   urlcolor=blue,           % color of external links
           pdfhighlight =/O]{hyperref}																		% dvipdfm package pour lien hypertextes dans pdf (pdfhighlight--> afficher la main dans le pdf, colorlinks--> colorlinks=true afficher les liens en couleurs)
  
\usepackage{blindtext}
\usepackage[final]{pdfpages}
\usepackage[french]{cleveref}																				% à placer après hyperref attention ne pas utiliser ":" dans les labels des équations avec French babel activé et cref

\usepackage{fancyhdr}
%%%%%%%%%%%%%%%%%%%%%%%%%%%%%%%%%%%%%%%%%%%%%%%%%%%%%%%%%%%%%%%%%%%%%%%%%%
%                         FORMAT PERSONNALISES                           %
% %%%%%%%%%%%%%%%%%%%%%%%%%%%%%%%%%%%%%%%%%%%%%%%%%%%%%%%%%%%%%%%%%%%%%%%%

\parindent=0pt        %leading space for paragraphs
\pagestyle{fancy}
\renewcommand{\arraystretch}{1.5}
\renewcommand{\headrulewidth}{0pt}
\fancyhead[CE,CO,LE,LO,RE,RO]{} %% clear out all headers
\fancyhead[C]{%
\begin{tabular}{|m{3.0cm}|m{10cm}|c@{}|}
\hline
\multirow{2}{*}{\includegraphics[scale=0.25]																			% LOGO LABO
{logo1.jpg}}& \centering \multirow{2}{*}{ \Large{\thetitle}}  &  \multirow{2}{*}{Page: \thepage ~/ \pageref{LastPage}}\\
& &  \\
\hline
%Etabli par: \newline \theauthor & \multicolumn{2}{l|}{Diffusion: interne}\\
%\hline
\multicolumn{2}{|l|}{Objet: \thetitleobject }& Date: \headertoday ~~\\ 
\hline
\end{tabular}
}
\renewcommand\footrulewidth{1pt}
\fancyfoot[L]{Document confidentiel} 																		% DOCUMENT CONFIDENTIEL
\fancyfoot[R]{Laboratoire SYMME}		 																	% NOM DU LABO

\newcolumntype{x}[1]{>{\centering\hspace{0pt}}p{#1}}
\setlength{\doublerulesep}{\arrayrulewidth} 


\usepackage{tikz}																							% pour les diagrammes
\usetikzlibrary{calc}
\usetikzlibrary{shapes.geometric,shapes.arrows,decorations.pathmorphing}
\usetikzlibrary{matrix,chains,scopes,positioning,shapes,arrows,fit}
\usetikzlibrary{calc,decorations.pathreplacing}
\usetikzlibrary{calc}
\usetikzlibrary{backgrounds,decorations.markings}

\setcounter{secnumdepth}{5}		%sous sous sous section



%%%%%%%%%%%%%%%%%%%%%%%%%%%%%%%%%%%%%%%%%%%%%%%%%%%%%%%%%%%%%%%%%%%%%%%%%%%%%%%%%%%%
%%-------------> PAGE DE GARDE INFO
%%%%%%%%%%%%%%%%%%%%%%%%%%%%%%%%%%%%%%%%%%%%%%%%%%%%%%%%%%%%%%%%%%%%%%%%%%%%%%%%%%%%

\author{Auteur}
\newcommand{\validator}{M Zaghdoudi}
\title{Rapport d'avancement}
\selectlanguage{french}	
\date{\today}
\newcommand{\thetitleobject}{Tarification santé : Indexation et mutualisation }
\setcounter{tocdepth}{6}
\setcounter{secnumdepth}{6}


%%%%%%%%%%%%%%%%%%%%%%%%%%%%%%%%%%%%%%%%%%%%%%%%%%%%%%%%%%%%%%%%%%%%%%%%%%%%%%%%%%%%
%%-------------> DEBUT DU DOCUMENT 
%%%%%%%%%%%%%%%%%%%%%%%%%%%%%%%%%%%%%%%%%%%%%%%%%%%%%%%%%%%%%%%%%%%%%%%%%%%%%%%%%%%%

\begin{document}

\graphicspath{{Figures/}}

%%%%%%%%%%%%%%%%%%%%%%%%%%%%%%%%%%%%%%%%%%%%%%%%%%%%%%%%%%%%%%%%%%%%%%%%%%%%%%%%%%%%
%%-------------> PAGE DE GARDE 
%%%%%%%%%%%%%%%%%%%%%%%%%%%%%%%%%%%%%%%%%%%%%%%%%%%%%%%%%%%%%%%%%%%%%%%%%%%%%%%%%%%%
\begin{titlepage}
    \centering
    \vspace*{3cm}
    {\bfseries\Large
        \thetitle\\
        \thetitleobject\\
        --- \\
        Mongi Zaghdoudi\\  																			% LIGNE "CONFIDENTIEL"
        \vskip2cm
    }
    %\includegraphics[scale=0.65]{logos.png}  																% LOGOS DES PARTENAIRES
    % \vfill
    % \begin{center}
    % \begin{table}[b]
    % \begin{tabular}{x{.225\linewidth}|| x{.225\linewidth}|| x{.225\linewidth} || x{.225\linewidth} }
    % \hline \hline
    % \textbf{Date} & \textbf{Révision} & \textbf{Rédigé par} & \textbf{Validé par} \tabularnewline
    % \hline \hline
    % \thedate & Rev. A  &  \theauthor &  \validator \tabularnewline
    % \hline
    % ~ & ~ & ~ & ~ \tabularnewline
    % \hline
    % ~ & ~ & ~ & ~ \tabularnewline
    % \hline
    % ~ & ~ & ~ & ~\tabularnewline         
    % \hline \hline                                                                   
    % \end{tabular}
    % \end{table}
    % \end{center}
    % \vfill
    % \vfill
    % \begin{tikzpicture}[remember picture,overlay]
    % \node (label) at (8cm,20cm){
    %     \includegraphics[width=2.5cm]{Logo-Symme.png} 													% LOGO LABO
    %   };
    
	% \draw[very thick]
	% 	([yshift=-25pt,xshift=25pt]current page.north west)--
	% 	([yshift=-25pt,xshift=-25pt]current page.north east)--
	% 	([yshift=25pt,xshift=-25pt]current page.south east)--
	% 	([yshift=25pt,xshift=25pt]current page.south west)--cycle;
    % \end{tikzpicture}
\end{titlepage}


%\begin{titlepage}
%    \centering
%    
%    \vspace*{3cm}
%    {\bfseries\Large
%        \thetitle\\
%        \thetitleobject\\
%        \vskip2cm
%    }    
%    \vfill
%    \begin{center}
%    \begin{table}[b]
%    \begin{tabular}{x{.225\linewidth}|| x{.225\linewidth}|| x{.225\linewidth} || x{.225\linewidth} }
%    \hline \hline
%    \textbf{Date} & \textbf{R�vision} & \textbf{R�dig� par} & \textbf{Valid� par} \tabularnewline
%    \hline \hline
%    \thedate & Rev. A  &  \theauthor &  \validator \tabularnewline
%    \hline
%    ~ & ~ & ~ & ~ \tabularnewline
%    \hline
%    ~ & ~ & ~ & ~ \tabularnewline
%    \hline
%    ~ & ~ & ~ & ~\tabularnewline         
%    \hline \hline                                                                   
%    \end{tabular}
%    \end{table}
%    \end{center}
%    \vfill
%    \vfill
%    \begin{tikzpicture}[overlay,remember picture]
%    \node (label) at (6.7cm,19.5cm){
%        \includegraphics[width=4cm]{../Figures/logo_CT1.pdf} % also works with logo.pdf
%      };
%    
%    \draw [line width=1pt,rounded corners=7pt]
%        ($ (current page.north west) + (1.5cm,-1.5cm) $)
%        rectangle
%        ($ (current page.south east) + (-1.5cm,1.5cm) $);
%        
%%	\draw[very thick]
%%		([yshift=-25pt,xshift=25pt]current page.north west)--
%%		([yshift=-25pt,xshift=-25pt]current page.north east)--
%%		([yshift=25pt,xshift=-25pt]current page.south east)--
%%		([yshift=25pt,xshift=25pt]current page.south west)--cycle;
%    \end{tikzpicture}
%\end{titlepage}


%%%%%%%%%%%%%%%%%%%%%%%%%%%%%%%%%%%%%%%%%%%%%%%%%%%%%%%%%%%%%%%%%%%%%%%%%%%%%%%%%%%%
%%-------------> SOMMAIRE
%%%%%%%%%%%%%%%%%%%%%%%%%%%%%%%%%%%%%%%%%%%%%%%%%%%%%%%%%%%%%%%%%%%%%%%%%%%%%%%%%%%%

\renewcommand\contentsname{Sommaire}
\setcounter{chapter}{1}
\tableofcontents
%\listoffigures
%\listoftables


%%%%%%%%%%%%%%%%%%%%%%%%%%%%%%%%%%%%%%%%%%%%%%%%%%%%%%%%%%%%%%%%%%%%%%%%%%%%%%%%%%%%
%%-------------> CORPS DOCUMENT
%%%%%%%%%%%%%%%%%%%%%%%%%%%%%%%%%%%%%%%%%%%%%%%%%%%%%%%%%%%%%%%%%%%%%%%%%%%%%%%%%%%%

%-----------------------------------------------------------------------------------
%-----------------------------------------------------------------------------------
\newpage

\section{Introduction}
\subsection{Besoins et contexte}
Dans le cadre du renouvellement annuel et afin de prendre en considération les hausses de prix 
dans les différents secteurs liés à la consommation de Biens et Service médicaux, la mutuelle VEBCN
 lance un projet d'indexation des tarifs en se basant sur les l'inflation observée dans différents 
 secteurs ainsi que l'évolution du PASS (Plafond Annuel de la Sécurité Sociale)

% \begin{figure}[hbtp]
% 	\centering
% 	\def\svgwidth{0.9\columnwidth}
% 	\fontsize{10pt}{10pt}\selectfont\input{Figures/assistant_tableau.pdf_tex}
% 	\caption{Assistant Texmaker}
% 	\label{figure_assistant_tableau}
% \end{figure}

\subsection{Outils et techniques}
Le travail réalisé, bien qu'il utilise une base fictive est basé sur des techniques actuarielles 
et des calculs mathématiques de la tarification santé.
Les outils et techniques utilisés sont les suivants : 
Cet outil en ligne permet par exemple de :
\begin{itemize}
\item Altair analytics workbench : Logiciel pour interprétation du SAS
\item Langage SAS : pour les calculs 
\item Langage \LaTeX() : pour l'édition du présnet rapport
\item Github : pour le partage des travaux 
\item VSCode comme éditeur et interpréteur \LaTeX()
\item Microsoft Word : pour la génération du courrier de publipostage
\item Microsoft Excel : pour les tables de données d'entrée et les exports 
\item \ldots
\end{itemize}



\section{Introduction}
\subsection{Besoins et contexte}
% \begin{figure}[hbtp]
% 	\centering
% 	\def\svgwidth{1\columnwidth}
% 	\fontsize{10pt}{10pt}\selectfont\input{Figures/generateur.pdf_tex}
% 	\caption{Générateur en ligne}
% 	\label{figure_generateur_tableau}
% \end{figure}

Cet outil en ligne permet par exemple de :
\begin{itemize}
\item Copier/Coller des tableaux Excel/Word et de générer le code associé;
\item Générer son propre tableau;
\item Gérer la fusion des cellules;
\item Gérer les alignements;
\item Gérer les bordures;
\item \ldots
\end{itemize}


\section{Exemples}

Un exemple de tableau simple (Tableau~\ref{tableau_simple}) et un exemple de tableau complexe (Tableau~\ref{tableau_complexe}).
\begin{table}[hbtp]
\centering
\begin{tabular}{|c|c|c|}
\hline
A & B & C \\ \hline
1 & 2 & 3 \\ \hline
4 & 5 & 6 \\ \hline
\end{tabular}
\caption{Tableau simple}
\label{tableau_simple}
\end{table}


\begin{table}[hbtp]
\centering
\begin{tabular}{|c|c|c|c|c|}
\hline
\multicolumn{2}{|c|}{\multirow{2}{*}{}} & \multicolumn{3}{c|}{E} \\ \cline{3-5} 
\multicolumn{2}{|c|}{}                  & a      & b     & c     \\ \hline
\multirow{3}{*}{F}          & t         & 1      & 2     & 3     \\ \cline{2-5} 
                            & u         & 4      & 5     & 6     \\ \cline{2-5} 
                            & v         & 7      & 8     & 9     \\ \hline
\end{tabular}
\caption{Tableau complexe avec cellules fusionnées}
\label{tableau_complexe}
\end{table}

%%%%%%%%%%%%%%%%%%%%%%%%%%%%%%%%%%%%%%%%%%%%%%%%%%%%%%%%%%%%%%%%%%%%%%%%%%%%%%%%%%%%
%%-------------> FIN DU DOCUMENT
%%%%%%%%%%%%%%%%%%%%%%%%%%%%%%%%%%%%%%%%%%%%%%%%%%%%%%%%%%%%%%%%%%%%%%%%%%%%%%%%%%%%

\end{document}